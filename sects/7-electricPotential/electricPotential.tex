\section{Electric Potential}
$V=-\frac{W_{\:}}{q}$\\
$V=\frac{-W_{\infty }}{q_0}=\frac{U}{q_0}$ (volt=joule per coulomb)\\
\subsection{Units}
$1N/C=\left(1\frac{N}{C}\right)\left(\frac{1V\cdot C}{1J}\right)\left(\frac{1J}{1N\cdot m}\right)=1V/m$ ;
$1eV=e\left(1 V\right)=\left(1.60\times 10^{-19} C\right)\left(1 J/C\right)=1.60\times 10^{-19}J$\\
\subsection{Work done by Applied Force}
$\Delta K=K_f-K_i=W_{appF}+W_{efF}$\\
$W_{appF}=-W_{efF}$ if stationary before and after move $K_f=K_i=0$\\
$\Delta U=U_f-U_i=W_{appF}$\\
$\Delta U=U_f-U_i=W_{appF}$\\
\subsection{Equipotential Surfaces}
$V_f-V_i=-\int _i^f\vec{E}\cdot d\vec{s}\:$\\
$V=-\int _i^f\vec{E}\cdot \:d\vec{s}\:$ se $V_i = 0$ ex:infinity\\
\subsection{Potential due to Point Charge}
$V=\frac{1}{4\pi \varepsilon _0}\:\frac{q}{r}$\\
\subsection{Potential due to GroupOfPointCharges}
Superposition principle, the sum of the values, unlike electric fields that was a sum of vectors
$V=\sum _{i=1}^n\left(V_i\right)=\frac{1}{4\pi \varepsilon _0}\:\sum _{i=1}^n\left(\frac{q_i}{r_i}\right)$\\
\subsection{Potential due to Electric Dipole}
$\vec{p}=qd$\\
if $r>>d$\\
$V=\frac{1}{4\pi \varepsilon _0}\:\frac{p\:cos\theta }{r^2}$\\
\subsection{Potential due to ContinousDistribution}
> $V=\frac{1}{4\pi \varepsilon _0}\int \frac{dq}{r}=\frac{1}{4\pi \varepsilon _0}\int \frac{\lambda dx}{r}$\\
\subsection{Calculating field from potential}
> $E_s=-\frac{\partial V}{\partial s}$\\
> $E_x=-\frac{\partial V}{\partial x}\:;\:E_y=-\frac{\partial \:V}{\partial \:y}\:;\:E_z=-\frac{\partial \:V}{\partial \:z}$\\
> $E=-\frac{\Delta V}{\Delta s}$\\
\subsection{ElectricPotentialEnergy SystemPointCharges}
$U=W=q_2V=\frac{1}{4\pi \varepsilon _0}\:\frac{q_1q_2}{r}$\\
