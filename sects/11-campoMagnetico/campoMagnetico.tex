\section{Campo Magnetico}
$\vec{F_B}=q\left(\vec{v}\times \vec{B}\right)$\\
$F_B=\left|q\right|vBsen\left(\phi \right)$\\
Regra mao direita(pistola): polegar $F_B$, indicador v, palma B\\
Unidade campo magnetico: Testla(T) = $T=\frac{Wb}{m^2}=\frac{N}{C\cdot \left(m/s\right)}=\frac{N}{A\cdot m}$ (Wb é weber) (Unidade cgs é um gauss(G) 1T=$10^4$G)\\
\subsection{ForcaParticCarregMovCircular}
$\left|q\right|vB=\frac{mv^2}{r}$\\
$r=\frac{mv}{\left|q\right|B}$\\
$\omega =\frac{v}{r}=\frac{\left|q\right|B}{m}$\\
$f=\frac{1}{T}=\frac{\omega \:}{2\pi \:}=\frac{\left|q\right|B}{2\pi \:m}$\\
\subsection{MovParticCarregCampoElecMag}
> $\vec{F}=q\vec{E}+q\left(\vec{v}\times \vec{B}\right)$\\
\subsection{SelectorVelocidade}
> Se $F_B$ = $F_E$ : $v=\frac{E}{B}$ mov reto\\
\subsection{CMag devido CorrenteRectilinea}
Regra mao direita: polegar corrente eletrica, dedos dobram no sentido de B ;
Pontos(sairPag, cima) sao dados pela ponta dos dedos, do lado q aparecem primeiro dps de dar volta fio, outro lado cruz(entrarPag, baixo) ;
Fio empurrado po lado do campo contario ao q esta a receber\\
\subsection{CMag Exerc Forca Corrent}
$\vec{F_B}=q\left(\vec{v}\times \:\vec{B}\right)nAL$ n numero de cargas, A area, L comprimento\\
$\vec{F}=i\vec{L}\times \vec{B}\:$ força sobre um fio retilineo\\
$\vec{F}=i\int _a^bd\vec{L}\times \vec{B}\:$ força sobre um fio\\
\subsection{CMag Exerc MomentForca Espira}
$\tau =iABsin\left(\theta \right)$ , A area , N num espiras, $\theta$ angl normalA e B (max qd espira paralelo campo)\\
$\vec{\tau }=Ni\vec{A}\times \vec{B}$ , A intensidade area, sentido normalDeA\\
Regra mao direita espiras (achar sentido de A): Dedos curvam sentido da corrente, polegar sentido de A\\
\subsection{Momento Dipolar Magnetico}
$\vec{\mu }=Ni\vec{A}$\\
$\vec{\tau }=\vec{\mu }\times \vec{B}$ momento de Forca\\
$U\left(\theta \right)=-\vec{\mu }\times \:\vec{B}$ energia potencial magentica\\
\subsection{CMag Prodz por Dipolo}
$\vec{B}\left(z\right)\cong \frac{\mu _0}{2\pi }\:\frac{\vec{\mu }}{z^3}$\\
com $\mu _0=4\pi \times 10^{-7}\:T\cdot m/A$\\