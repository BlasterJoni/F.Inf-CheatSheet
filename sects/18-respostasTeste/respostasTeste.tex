\section{oh oh stinky}
$
1. Numa experiência de interferência a duas fendas, é-lhe pedido que utilize luz laser de diferentes comprimentos de onda e determine a separação entre máximos adjacentes. E consegue observar que a separação é maior quando ilumina as fendas com: B) luz vermelha
2. Um eletrão move-se no sentido do eixo +x e entra numa região onde existe um campo magnético. Se o eletrão ficar sujeito a uma deflexão segundo -y, o campo magnético aponta segundo o eixo: C) -z
3. Um diagrama de difração é produzido num alvo que dista 1,30m de uma única fenda. Se for utilizada luz de ??(630 ou 670 ou 60) mm e a distância do centro da banda central brilhante até à primeira banda escura for 5,3*10^-3 m, então a largura da fenda vale: E) 1,52*10^-4 m
4. A luz de 532 mm de um laser passa através de uma abertura circular. É observada num alvo situado a 4,0 m para lá da abertura. A largura do máximo central é de 1,1 cm. Quanto vale o diâmetro do orifício? B) 472 μm
5. Uma partícula carregada é injetada num campo magnético uniforme de modo que o seu vetor de velocidade é perpendicular às linhas de campo magnético. Ignorando o peso da partícula, esta irá: B) fazer um caminho circular
6. Um eletrão tem uma velocidade horizontal (nesta direção e sentido: → ) de 10 km/s e entra numa zona em que existe um campo magnético de um tesla "para fora da folha". A força a que fica sujeito é: D) para cima (i.e. ↓) e vale ≅ 1,6*10^-15 N
7. Numa experiência a duas fendas, a separação das fendas é de 3,00*10^-5 m. O diagrama de interferência é gravado num detetor plano situado a 2,00 m das fendas. Se a 7ª risca brilhante no detetor estiver a 10,0 ??(cm ou m) da risca central, quanto valerá o comprimento da luz que incide nas fendas? B) 214 nm
8. Seja uma linha de transmissão RC de comprimento L e apresentado um atraso δ. Se aumentarmos a linha para um comprimento 2L, o modelo de Elmore dá como estimativa para o novo atraso: C) 4δ
9. Uma bobina transporta a corrente I(t) = (0,500A)cos[(275s^-1)t]. Se a máxima fem na bobina for igual a 0,500V, quanto valerá a auto-indutância da bobina? E) 3,64 mH
10. Na figura, um fio e uma resistência de 10 ohm são usadas para formar um circuito na forma de uma quadrado de 20 cm de lado. Um campo magnético uniforme mas não estacionário é dirigido como se mostra na figura. A intensidade do campo magnético decresce de 1,20T para 0,40T no intervalo de 55 ms. A corrente induzida média e o seu sentido através da resistência, nesse intervalo de tempo, é cerca de: D) 58 mA, de b para a
11. Um condutor fechado formando uma espira circular de 2,0 m de raio está localizado num campo magnético uniforme mas variável. Se a fem induzida máxima na espira for de 5,0 V, qual é a taxa máxima a que a intensidade do campo magnético está a variar se o campo magnético estiver orientado perpendicularmente ao plano da espira? C) 0,40 T/s
12. Um fio, no plano da folha, transporta uma corrente no sentido da parte inferior da folha, como se mostra na figura. Qual a direção e sentido da força magnética , que é executada num eletrão que se aproxima do fio representado na figura: C) diretamente dirigido para a parte inferior da folha
13. Um varão condutor de 25 cm de comprimento é colocado num cabo metálico em forma de U, com a resistência R de 8 Ω como se mostra na figura. O cabo e o varão estão no plano da folha de papel. Um campo magnético constante de 0,4T de intensidade é aplicado perpendicularmente (e a entrar no papel). Uma força aplicada ao varão move-o para a direita com uma velocidade constante de 6 m/s. Qual é a intensidade da fem induzida no cabo? A) 0,6 V
14. Quando um material ferromagnético é colocado num campo magnético exterior, o campo magnético resultante dos seus domínios magnéticos torna-se: E) superior
15. Para uma dada onda eletromagnética em espaço livre, num determinado instante o vetor de campo elétrico aponta segundo +z enquanto o vetor de campo magnético aponta segundo +x, como se mostra na figura. Em que direção e sentido se desloca a onde? A) +y
16. Um fio horizontal transporta uma corrente dirigida diretamente para si. Do seu ponto de vista, o campo magnético causado por esta corrente: B) faz círculos no sentido contrário ao relógio em torno do fio
17. Se a luz não polarizada de intensidade Io passar através de um polarizador ideal, qual a intensidade da luz emergente? D) Io/2
18. Para uma onda eletromagnética em espaço livre com um campo elétrico de amplitude E e um campo magnético de amplitude B, a razão B/E é igual a: A) 1/c
19. Num transformador ideal a razão entre o número de espiras do enrolamento secundário e primário, Ns/Np, é igual a 100. Se a amplitude da corrente primária for √2 A, qual é o valor eficaz da corrente no secundário? A) 1/100 A
20. Uma corrente alternada é fornecida a um componente elétrico que tem a especificação que só pode ser utilizado para tensões abaixo de 16 V. Qual é o valor mais elevado de Veficaz que poderá ser fornecido de modo a que aquele componente funcione em segurança? C) 8√2 V
$